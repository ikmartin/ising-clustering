\documentclass{article}
\usepackage{geometry}[1in]
\usepackage{amsmath}
\usepackage{amsfonts}

\newcommand{\R}{{\mathbb{R}}}
\newcommand{\Z}{{\mathbb{Z}}}
\DeclareMathOperator*{\argmax}{arg\,max}
\DeclareMathOperator*{\argmin}{arg\,min}
\DeclareMathOperator*{\softmax}{soft\,max}

\title{Quadratic Rigidity}
\author{Andrew Moore}
\date{Summer 2023}
\begin{document}
\maketitle

The goal is to figure out what number and distribution of constraints we need on a given input level in order to get at least a certain probability that the entire constraint matrix will be satisfied (or really, satisfiable, but that is a harder problem it seems). 

One way to attack this is to take advantage of the rigidity of the quadratic function class. Suppose that $H$ is a quadratic polynomial in $x_1 \dots x_n$, where we have coefficients $c_{ij} \sim N(0,1)$. 



\end{document}
