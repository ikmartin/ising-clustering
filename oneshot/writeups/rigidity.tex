\documentclass{article}
\usepackage{geometry}[1in]
\usepackage{amsmath}
\usepackage{amsfonts}

\newcommand{\R}{{\mathbb{R}}}
\newcommand{\Z}{{\mathbb{Z}}}
\DeclareMathOperator*{\argmax}{arg\,max}
\DeclareMathOperator*{\argmin}{arg\,min}
\DeclareMathOperator*{\softmax}{soft\,max}

\title{Quadratic Rigidity}
\author{Andrew Moore}
\date{Summer 2023}
\begin{document}
\maketitle

The goal is to figure out what number and distribution of constraints we need on a given input level in order to get at least a certain probability that the entire constraint matrix will be satisfied (or really, satisfiable, but that is a harder problem it seems). 

One way to attack this is to take advantage of the rigidity of the quadratic function class. Suppose that $H$ is a quadratic polynomial in $x_1 \dots x_n$, where we have coefficients $c_{ij} \sim N(0,1)$. 

\section{Building}

Suppose that we have two quadratic Hamiltonians $H_1(x, y, o_1, a_1)$ and $H_2(x, y, o_1, o_2, a_2)$ such that if $M = 2\max |H_2|+1/2$, then
\begin{align}
	H_1(x, y, o_1, a_1) <= H_1(x, y, o_1', a_1') - M &&\forall a_1' \forall o_1' \neq o_1\\
	H_2(x,y,o_1,o_2, a_2) <= H_2(x,y,o_1,o_2',a_2')-1/2 &&\forall a_2' \forall o_2' \neq o_2
\end{align}
Then we can make a new Hamiltonian $H(x,y,o_1,o_2,a_1,a_2) = H_1(x,y,o_1,a_1) + H_2(x,y,o_1,o_2,a_2)$. Select wrong outputs $o_1', o_2'$ and arbitrary auxes $a_1', a_2'$. 
\begin{align}
	H(x,y,o_1',o_2',a_1',a_2') = H_1(x,y,o_1',a_1') + H_2(x,y,o_1',o_2',a_2') \\
	\geq H_1(x,y,o_1,a_1) + M + H_2(x,y,o_1,o_2',a_2') + (H_2(x,y,o_1', o_2', a_2') - H_2(x,y,o_1,o_2',a_2'))
\end{align}
Note that $H_2(x,y,o_1', o_2', a_2') - H_2(x,y,o_1,o_2',a_2') \geq -2\max|H_2|$, so we can continue:
\begin{align}
	\geq H_1(x,y,o_1,a_1) + 1/2 + H_2(x,y,o_1,o_2',a_2') \geq H_1(x,y,o_1,a_1) + H_2(x,y,o_1,o_2,a_2) + 1\\
	= H(x,y,o_1,o_2,a_1,a_2) + 1
\end{align}
This shows that $H$ satisfies the constraint set. 




\end{document}
