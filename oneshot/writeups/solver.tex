\documentclass{article}
\usepackage{geometry}[1in]
\usepackage{amsmath}
\usepackage{amsfonts}

\newcommand{\R}{{\mathbb{R}}}
\newcommand{\Z}{{\mathbb{Z}}}
\DeclareMathOperator*{\argmax}{arg\,max}
\DeclareMathOperator*{\argmin}{arg\,min}
\DeclareMathOperator*{\softmax}{soft\,max}

\title{Solver Documentation}
\author{Andrew Moore}
\date{Summer 2023}
\begin{document}
\maketitle

Our goal is to solve the linear programming problem $M\phi \geq v$, where in general $v$ is the vector of all ones. In practice we can reformulate the problem as 
\begin{align*}
		\min_{\phi, \rho}\ \langle 1, \rho\rangle &\text{ s.t. } M\phi + \rho \geq v, \rho \geq 0
\end{align*}
Note that this is equivalent to
\begin{align*}
		\max_{\lambda, s}\ \langle b, \lambda \rangle &\text{ s.t. } A^T \lambda + s = c, s \geq 0
\end{align*}
If we make the identification
\begin{align*}
		&b = \begin{bmatrix}
				0\\
				-1\end{bmatrix} 
		&c = \begin{bmatrix}
				-v\\
				0\end{bmatrix}
		&&A^T = \begin{bmatrix}
				-M & -I\\
				0  & -I
		\end{bmatrix}
\end{align*}
Since if $\lambda = (\lambda_1, \lambda_2)$ and $s = (s_1, s_2)$, then $A^T\lambda + s = c$ can be written as the pair of equations $-M\lambda_1 - \lambda_2 + s_1 = -v$ and $-\lambda_2 + s_2 = 0$, which can be re-arranged as $s_2 = \lambda_2$ and $s_1 = M\lambda_1 + \lambda_2 - v$, so $s \geq 0$ actually means $\lambda_2 \geq 0$ and $M\lambda_1 + \lambda_2 \geq v$. This recovers our original problem with the identification $\phi := \lambda_1$, $\rho := \lambda_2$. 

A word on the dimensions. Suppose that $M \in \R^{m \times n}$, with $m >> n$. Then $A^T \in \R^{2m \times n+m}$, so $\lambda, b \in \R^{n+m}$ and $s, c, x \in \R^{2m}$ 


Therefore, we can start plugging in these special forms to the calculations which must be made in the MCP:
\begin{align}
		r_b = Ax - b = \begin{bmatrix}
				-M^T & 0\\
				-I & -I
				\end{bmatrix} \begin{bmatrix}
				x_1\\
				x_2\end{bmatrix}
				- \begin{bmatrix}
						0\\
						-1
				\end{bmatrix}
				= \begin{bmatrix}
				-M^Tx_1\\
				 -x_1-x_2+1
				\end{bmatrix}\\
		r_c = A^T\lambda + s - c = \begin{bmatrix}
				-M & -I\\
				0  & -I
				\end{bmatrix} \begin{bmatrix} \lambda_1 \\ \lambda_2\end{bmatrix} + \begin{bmatrix} s_1 \\ s_2 \end{bmatrix} - \begin{bmatrix} -v \\ 0 \end{bmatrix} = 
				\begin{bmatrix}
						-M\lambda_1 - \lambda_2 + s_1 + v\\
						-\lambda_2 + s_2
				\end{bmatrix}
\end{align}
We also get a more detailed block form for the coefficient matrix:
\begin{align}
		C = \begin{bmatrix}
				0    & 0   & -M & -I & I   & 0\\
				0    & 0   &  0 & -I & 0   & I\\
				-M^T & 0   &  0 &  0 & 0   & 0\\
				-I   & -I  &  0 &  0 & 0   & 0\\
				S_1  &  0  &  0 &  0 & X_1 & 0\\
				0    & S_2 &  0 &  0 & 0   & X_2
		\end{bmatrix}
\end{align}
The predictor step is therefore the solution to the system of equations
\begin{align}
		\begin{bmatrix}
				0    & 0   & -M & -I & I   & 0\\
				0    & 0   &  0 & -I & 0   & I\\
				-M^T & 0   &  0 &  0 & 0   & 0\\
				-I   & -I  &  0 &  0 & 0   & 0\\
				S_1  &  0  &  0 &  0 & X_1 & 0\\
				0    & S_2 &  0 &  0 & 0   & X_2
		\end{bmatrix}\begin{bmatrix}
				\Delta x_1\\
				\Delta x_2\\
				\Delta \lambda_1\\
				\Delta \lambda_2\\
				\Delta s_1\\
				\Delta s_2\\
		\end{bmatrix}
		=\begin{bmatrix}
				M\lambda_1 + \lambda_2 - s_1 - v\\
				\lambda_2 - s_2\\
				M^Tx_1\\
				x_1 + x_2 - 1\\
				-x_1 * s_1\\
				-x_2 * s_2
		\end{bmatrix}
\end{align}
\begin{align}
		-M^T \Delta x_1 = M^T x_1\\
		-\Delta x_1 - \Delta x_2 = x_1 + x_2 - 1\\
		s_1 * \Delta x_1 + x_1 * \Delta s_1 = - x_1 * s_1\\
		s_2 * \Delta x_2 + x_2 * \Delta s_2 = - x_2 * s_2\\
		-\Delta \lambda_2 + \Delta s_2 = \lambda_2 - s_2\\
		-M\Delta \lambda_1 - \Delta \lambda_2 + \Delta s_1 = M\lambda_1 + \lambda_2 - s_1 - v
\end{align}
It appears that we may simply let $\Delta x_1 := -x_1$. Then
\begin{align}
		x_1 - \Delta x_2 = x_1 + x_2 - 1 \implies \Delta x_2 = 1 - x_2\\
		-s_1 * x_1 + x_1 * \Delta s_1 = -x_1 * s_1 \implies x_1 * \Delta s_1 = 0 \implies \Delta s_1 = 0\\
		s_2 * (1-x_2) + x_2 * \Delta s_2 = -x_2 * s_2 \implies x_2 * \Delta s_2 = -s_2
\end{align}




Switch columns 1-2 and 5-6, then switch rows 3-4 and 5-6:
\begin{align}
		\begin{bmatrix}
				I	&	0	&	-M	&	-I	&	0	&	0\\
				0	&	I	&	0	&	-I	&	0	&	0\\
				X_1	&	0	&	0	&	0	&	S_1	&	0\\
				0	&	X_2	&	0	&	0	&	0	&	S_2\\
				0	&	0	&	0	&	0	& -M^T	&	0\\
				0	&	0	&	0	&	0	&	-I	&	-I\\
		\end{bmatrix}
\end{align}

\begin{align}
		\begin{bmatrix}
				0 & A^T & I\\
				A & 0   & 0\\
				S & 0   & X
		\end{bmatrix}
		\begin{bmatrix}
				\Delta x\\
				\Delta \lambda\\
				\Delta s
		\end{bmatrix} 
		= 
		\begin{bmatrix}
				-r_c\\
				-r_b\\
				-x*s
		\end{bmatrix}\\
		\begin{bmatrix}
				I & A^T & 0\\
				0 & 0   & A\\
				X & 0   & S
		\end{bmatrix}
		\begin{bmatrix}
				\Delta s\\
				\Delta \lambda\\
				\Delta x
		\end{bmatrix}
		= \begin{bmatrix}
				-r_c\\
				-r_b\\
				-x*s
		\end{bmatrix}\\
		\begin{bmatrix}
				I & A^T & 0\\
				X & 0   & S\\
				0 & 0   & A
		\end{bmatrix}
		\begin{bmatrix}
				\Delta s\\
				\Delta \lambda\\
				\Delta x
		\end{bmatrix}
		= 
		\begin{bmatrix}
				-r_c\\
				-x*s\\
				-r_b
		\end{bmatrix}
\end{align}

Here is a transcription into mathematical language of Teresa's algorithm:
\begin{align*}
		\Delta s &\leftarrow (x * s - x * r_c) / s\\
		\Delta \lambda &\leftarrow A \Delta s - r_b\\
		\Delta x &\leftarrow x / s\\
		\Delta s &\leftarrow \Delta \lambda\\
		\Delta \lambda &\leftarrow \text{ solve system }(\Delta s, \Delta x)\\
		\Delta s &\leftarrow -A^T \Delta \lambda - r_c\\
		\Delta x &\leftarrow -(x*s + x * \Delta s) / s
\end{align*}
The constant reuse of variables is just for code optimization. Therefore, the actual function can be refactored into
\begin{align*}
		\Delta \lambda &\leftarrow \text{solve}(A((x*s - x*r_c) / s) - r_b, x/s)\\
		\Delta s &\leftarrow -A^T \Delta \lambda - r_c\\
		\Delta x &\leftarrow -(x*s + x*\Delta s) / s
\end{align*}

I still don't understand exactly what solve is doing. But maybe I can figure it out from context. We know that for the $\Delta$s to be a correct solution, $A^T\Delta \lambda + \Delta s = -r_c$, i.e.
\begin{align}
		\begin{bmatrix}
				-M & -I\\
				0 & -I
		\end{bmatrix}\begin{bmatrix}
				\Delta \lambda_1\\
				\Delta \lambda_2
		\end{bmatrix} 
		= \begin{bmatrix}
				M\lambda_1 + \lambda_2 - s_1 - v - \Delta s_1\\
				\lambda_2 - s_2 - \Delta s_2
		\end{bmatrix}
\end{align}
This is the equation used to compute $\Delta s$. Similarly, row 3 of the system is used to compute $\Delta x$. So we know that whatever we pick $\Delta \lambda$ to be, we will get $\Delta s = -r_c - A^T\Delta \lambda$ and hence $\Delta x = (x * (r_c + A^T\Delta \lambda) - x*s) / s$. But we need it to also be true that $A\Delta x = -r_b$. So we need to pick $\Delta \lambda$ such that 
\begin{align*}
		A((x*(r_c + A^T\Delta \lambda) - x*s)/s) = -r_b\\
		A(S^{-1}X(r_c + A^T\Delta \lambda) - x) = -r_b\\
		AS^{-1}Xr_c + AS^{-1}XA^T\Delta \lambda - Ax = -r_b\\
		AS^{-1}XA^T\Delta \lambda = Ax - r_b - AS^{-1}Xr_c\\
		AS^{-1}XA^T \Delta \lambda = 
		\begin{bmatrix}
				-m^t & 0\\
				-i & -i
		\end{bmatrix}
		\begin{bmatrix}
				x_1/s_1 & 0\\
				0 & x_2/s_2
		\end{bmatrix}
		\begin{bmatrix}
				-m & -i\\
				0 & -i
		\end{bmatrix}
		\begin{bmatrix}
				\delta \lambda_1\\
				\delta \lambda_2
		\end{bmatrix}
		\\= 
		\begin{bmatrix}
				M^T x_1\\
				x_1 + x_2 - 1
		\end{bmatrix}
		+ \begin{bmatrix}
				-M^T & 0\\
				-I & -I
		\end{bmatrix}
		\begin{bmatrix}	
				x_1 + (x_1/s_1) * (M\lambda_1 + \lambda_2 - s_1 - v)\\
				x_2 + (x_2/s_2) * (\lambda_2 - s_2)
		\end{bmatrix}
\end{align*}

Now, we need to investigate how SolveSysMainReward works. The arguments are rhs1, rhs2, y1, y2, d21, d22, as well as some other metadata and data storage arrays. 

Ok, I think I've got it.

We need to solve systems of the form
\begin{align}
		\begin{bmatrix}
				0 & A^T & I\\
				A & 0   & 0\\
				S & 0   & X
		\end{bmatrix}
		\begin{bmatrix}
				\Delta x\\
				\Delta \lambda\\
				\Delta s
		\end{bmatrix} 
		= 
		\begin{bmatrix}
				-r_c\\
				-r_b\\
				L
		\end{bmatrix}
\end{align}
Which, written out as equations, is
\begin{align}
		A^T\Delta \lambda + \Delta s = -r_c\\
		S\Delta x + X\Delta s = L\\
		A\Delta x = -r_b
\end{align}
We can re-arrange to determine that $\Delta s = -r_c - A^T\Delta \lambda$ and $\Delta x = S^{-1}(L - X\Delta s) = S^{-1}(L - X(-r_c - A^T\Delta \lambda))$, upon which the last equation becomes
\begin{align}
		AS^{-1}(L - X(-r_c - A^T\Delta \lambda)) = -r_b\\
		AS^{-1}L + AS^{-1}Xr_c + AS^{-1}XA^T\Delta \lambda = -r_b\\
		(AS^{-1}XA^T) \Delta \lambda = -r_b - AS^{-1}(L + Xr_c)\\
		(AS^{-1}XA^T) \Delta \lambda = -Ax + b - AS^{-1}(L + Xr_c)\\
		= b - A(x + S^{-1}(L + Xr_c))
\end{align}
This is now a much smaller linear system, but we can break it down further by working out the coefficient matrix:
\begin{align}
		AS^{-1}XA^T = 
		\begin{bmatrix}
				-M^T & 0\\
				-I & -I
		\end{bmatrix}
		\begin{bmatrix}
				S_1^{-1}X_1 & 0\\
				0 & S_2^{-1}X_2
		\end{bmatrix}
		\begin{bmatrix}
				-M & -I\\
				0 & -I
		\end{bmatrix}\\
		= 
		\begin{bmatrix}
				-M^T & 0\\
				-I & -I
		\end{bmatrix}
		\begin{bmatrix}
				-S_1^{-1}X_1M & -S_1^{-1}X_1\\
				0 & -S_2^{-1} X_2
		\end{bmatrix}
		= 
		\begin{bmatrix}
				M^TS_1^{-1}X_1M & M^TS_1^{-1}X_1\\
				S_1^{-1} X_1 M & S_1^{-1}X_1 + S_2^{-1}X_2
		\end{bmatrix}
\end{align}
Now, we can apply block UDL-decomposition. It will be convenient if the matrix that we have to invert is actually the bottom right (since it is diagonal), and therefore we need the formula 
\begin{align}
		\begin{bmatrix}
				W & X\\
				Y & Z
		\end{bmatrix}
		=
		\begin{bmatrix}
				I & XZ^{-1}\\
				0 & I
		\end{bmatrix}
		\begin{bmatrix}
				W - XZ^{-1}Y & 0\\
				0 & Z
		\end{bmatrix}
		\begin{bmatrix}
				I & 0\\
				Z^{-1}Y & I
		\end{bmatrix}
\end{align}
Applying this, we obtain (with $D := S_1^{-1}X_1$ and $Z := S_1^{-1}X_1 + S_2^{-1}X_2$)
\begin{align}
		\begin{bmatrix}
				I & M^T DZ^{-1}\\
				0 & I
		\end{bmatrix}
		\begin{bmatrix}
				M^T(D - D^2Z^{-1})M & 0\\
				0 & Z
		\end{bmatrix}
		\begin{bmatrix}
				I & 0\\
				DZ^{-1} M & I
		\end{bmatrix}\\
		= 
		\begin{bmatrix}
				I & M^T DZ^{-1}\\
				0 & I
		\end{bmatrix}
		\begin{bmatrix}
			M^T(D-D^2Z^{-1})M & 0\\
			DM & Z
		\end{bmatrix}
\end{align}
Now, consider the equation
\begin{align}
		\begin{bmatrix}
				I & M^T DZ^{-1}\\
				0 & I
		\end{bmatrix}
		\begin{bmatrix}
			y_1\\
			y_2
		\end{bmatrix}
		=
		\begin{bmatrix}
			u_1\\
			u_2
		\end{bmatrix}
\end{align}
We get $y_2 = u_2$ and $u_1 = y_1 + M^TDZ^{-1} y_2 = y_1 + M^TDZ^{-1} u_2$, so $y_1 = u_1 - M^TDZ^{-1}u_2$. Now, we want to solve
\begin{align}
		\begin{bmatrix}
			M^T(D-D^2Z^{-1})M & 0\\
			DM & Z
		\end{bmatrix}
		\begin{bmatrix}
			\Delta \lambda_1\\
			\Delta \lambda_2
		\end{bmatrix}
		=
		\begin{bmatrix}
			u_1 - M^TDZ^{-1}u_2\\
			u_2
		\end{bmatrix}
\end{align}
Clearly $DM \Delta \lambda_1 + Z\Delta \lambda_2 = u_2$ implies $\Delta \lambda_2 = Z^{-1}(u_2 - DM\Delta\lambda_1)$, so we have reduced the problem to solving the linear system $(M^T(D-D^2Z^{-1})M)\Delta \lambda_1 = u_1 - M^TDZ^{-1}u_2$. Since $n$ is in general fairly small, this is actually an easy system to solve. This leads us to the following algorithm:
\begin{align}
	r_c &\leftarrow A^T \lambda + s - c\\
	u &\leftarrow b - A(x + S^{-1}(Xr_c + L))\\
	\Delta \lambda_1 &\leftarrow (M^T(D-D^2Z^{-1})M)^{-1}(u_1 - M^TDZ^{-1}u_2)\\
	\Delta \lambda_2 &\leftarrow Z^{-1}(u_2 - DM\Delta\lambda_1) \\
	\Delta s &\leftarrow -r_c - A^T\Delta \lambda\\
	\Delta x &\leftarrow S^{-1}(L - X\Delta s)\\
\end{align}
\end{document}

