\documentclass{article}
\usepackage{geometry}[1in]
\usepackage{amsmath}
\usepackage{amsfonts}

\newcommand{\R}{{\mathbb{R}}}
\newcommand{\Z}{{\mathbb{Z}}}
\DeclareMathOperator*{\argmax}{arg\,max}
\DeclareMathOperator*{\argmin}{arg\,min}
\DeclareMathOperator*{\softmax}{soft\,max}

\title{Solver Documentation}
\author{Andrew Moore}
\date{Summer 2023}
\begin{document}
\maketitle

\section{Problem Setup}

Our goal is to solve the linear programming problem $M\phi \geq v$, where in general $v$ is the vector of all ones. In practice we can reformulate the problem as 
\begin{align*}
		\min_{\phi, \rho}\ \langle 1, \rho\rangle &\text{ s.t. } M\phi + \rho \geq v, \rho \geq 0
\end{align*}
Note that this is equivalent to
\begin{align*}
		\max_{\lambda, s}\ \langle b, \lambda \rangle &\text{ s.t. } A^T \lambda + s = c, s \geq 0
\end{align*}
If we make the identification
\begin{align*}
		&b = \begin{bmatrix}
				0\\
				-1\end{bmatrix} 
		&c = \begin{bmatrix}
				-v\\
				0\end{bmatrix}
		&&A^T = \begin{bmatrix}
				-M & -I\\
				0  & -I
		\end{bmatrix}
\end{align*}
Since if $\lambda = (\lambda_1, \lambda_2)$ and $s = (s_1, s_2)$, then $A^T\lambda + s = c$ can be written as the pair of equations $-M\lambda_1 - \lambda_2 + s_1 = -v$ and $-\lambda_2 + s_2 = 0$, which can be re-arranged as $s_2 = \lambda_2$ and $s_1 = M\lambda_1 + \lambda_2 - v$, so $s \geq 0$ actually means $\lambda_2 \geq 0$ and $M\lambda_1 + \lambda_2 \geq v$. This recovers our original problem with the identification $\phi := \lambda_1$, $\rho := \lambda_2$. A word on the dimensions. Suppose that $M \in \R^{m \times n}$, with $m >> n$. Then $A^T \in \R^{2m \times n+m}$, so $\lambda, b \in \R^{n+m}$ and $s, c, x \in \R^{2m}$. We can also go ahead and set $r_b \leftarrow Ax - b$ and $r_c \leftarrow A^T\lambda + s - c$. 

\section{Solving Systems of the Form $(AKA^T)p=q$}

Let $A$ be as defined above and $K$ be some general diagonal matrix with diagonal vector $k$ split into $k_1, k_2 \in \R^m$. The only actual systems of equations that we need to solve in this algorithm will be of the form $(AKA^T)p = q$. We will derive a general algorithm for solving this system in an efficient manner by taking advantage of the structure of $A$. Note that $p,q \in \R^{n+m}$.
\begin{align}
		AKA^T = 
		\begin{bmatrix}
				-M^T & 0\\
				-I & -I
		\end{bmatrix}
		\begin{bmatrix}
				K_1 & 0\\
				0 & K_2
		\end{bmatrix}
		\begin{bmatrix}
				-M & -I\\
				0 & -I
		\end{bmatrix}\\
		= 
		\begin{bmatrix}
				-M^T & 0\\
				-I & -I
		\end{bmatrix}
		\begin{bmatrix}
				-K_1M & -K_1\\
				0 & -K_2
		\end{bmatrix}
		= 
		\begin{bmatrix}
				M^TK_1M & M^TK_1\\
				K_1M & K_1 + K_2
		\end{bmatrix}
\end{align}
Now, we can apply block UDL-decomposition. It will be convenient if the matrix that we have to invert is actually the bottom right (since it is diagonal), and therefore we need the formula 
\begin{align}
		\begin{bmatrix}
				W & X\\
				Y & Z
		\end{bmatrix}
		=
		\begin{bmatrix}
				I & XZ^{-1}\\
				0 & I
		\end{bmatrix}
		\begin{bmatrix}
				W - XZ^{-1}Y & 0\\
				0 & Z
		\end{bmatrix}
		\begin{bmatrix}
				I & 0\\
				Z^{-1}Y & I
		\end{bmatrix}
\end{align}
Applying this, we obtain (with $D := K_1$ and $Z := K_1 + K_2$)
\begin{align}
		\begin{bmatrix}
				I & M^T DZ^{-1}\\
				0 & I
		\end{bmatrix}
		\begin{bmatrix}
				M^T(D - D^2Z^{-1})M & 0\\
				0 & Z
		\end{bmatrix}
		\begin{bmatrix}
				I & 0\\
				DZ^{-1} M & I
		\end{bmatrix}\\
		= 
		\begin{bmatrix}
				I & M^T DZ^{-1}\\
				0 & I
		\end{bmatrix}
		\begin{bmatrix}
			M^T(D-D^2Z^{-1})M & 0\\
			DM & Z
		\end{bmatrix}
\end{align}
Now, consider the equation
\begin{align}
		\begin{bmatrix}
				I & M^T DZ^{-1}\\
				0 & I
		\end{bmatrix}
		\begin{bmatrix}
			y_1\\
			y_2
		\end{bmatrix}
		=
		\begin{bmatrix}
			q_1\\
			q_2
		\end{bmatrix}
\end{align}
We get $y_2 = q_2$ and $q_1 = y_1 + M^TDZ^{-1} y_2 = y_1 + M^TDZ^{-1} q_2$, so $y_1 = q_1 - M^TDZ^{-1}q_2$. Now, we want to solve
\begin{align}
		\begin{bmatrix}
			M^T(D-D^2Z^{-1})M & 0\\
			DM & Z
		\end{bmatrix}
		\begin{bmatrix}
			p_1\\
			p_2
		\end{bmatrix}
		=
		\begin{bmatrix}
			q_1 - M^TDZ^{-1}q_2\\
			q_2
		\end{bmatrix}
\end{align}
Clearly $DM p_1 + Zp_2 = q_2$ implies $p_2 = Z^{-1}(q_2 - DMp_1)$, so we have reduced the problem to solving the linear system $(M^T(D-D^2Z^{-1})M)p_1 = q_1 - M^TDZ^{-1}q_2$. Since $n$ is in general fairly small, this is actually an easy system to solve. This leads us to the following algorithm:
\begin{align}
	p_1 &\leftarrow (M^T(D-D^2Z^{-1})M)^{-1}(q_1 - M^TDZ^{-1}q_2)\\
	p_2 &\leftarrow Z^{-1}(q_2 - DMp_1)
\end{align}

\section{Solving the Main System}

We need to solve systems of the form
\begin{align}
		\begin{bmatrix}
				0 & A^T & I\\
				A & 0   & 0\\
				S & 0   & X
		\end{bmatrix}
		\begin{bmatrix}
				\Delta x\\
				\Delta \lambda\\
				\Delta s
		\end{bmatrix} 
		= 
		\begin{bmatrix}
				-r_c\\
				-r_b\\
				L
		\end{bmatrix}
\end{align}
Which, written out as equations, is
\begin{align}
		A^T\Delta \lambda + \Delta s = -r_c\\
		S\Delta x + X\Delta s = L\\
		A\Delta x = -r_b
\end{align}
We can re-arrange to determine that $\Delta s = -r_c - A^T\Delta \lambda$ and $\Delta x = S^{-1}(L - X\Delta s) = S^{-1}(L - X(-r_c - A^T\Delta \lambda))$, upon which the last equation becomes
\begin{align}
		AS^{-1}(L - X(-r_c - A^T\Delta \lambda)) = -r_b\\
		AS^{-1}L + AS^{-1}Xr_c + AS^{-1}XA^T\Delta \lambda = -r_b\\
		(AS^{-1}XA^T) \Delta \lambda = -r_b - AS^{-1}(L + Xr_c)\\
		(AS^{-1}XA^T) \Delta \lambda = -Ax + b - AS^{-1}(L + Xr_c)\\
		= b - A(x + S^{-1}(L + Xr_c))
\end{align}
This gives us the following algorithm:
\begin{align}
	\Delta \lambda &\leftarrow (AS^{-1}XA^T)^{-1}(b-A(x + S^{-1}(Xr_c + L)))\\
	\Delta s &\leftarrow -r_c - A^T\Delta \lambda\\
	\Delta x &\leftarrow S^{-1}(L - X\Delta s)
\end{align}

\section{Calculating the Initial Guess}

The initial guesses are calculated using the closed form of a least-squares problem. In other words, we need to calculate $(AA^T)^{-1}$. As with the previous linear system, the structure of $A$ can be exploited to make this simpler.
\begin{align}
		AA^T = \begin{bmatrix}
				-M^T & 0\\
				-I & -I
		\end{bmatrix}
		\begin{bmatrix}
				-M & -I\\
				0 & -I
				\end{bmatrix}
				= \begin{bmatrix}
								M^TM & M^T\\
								M & 2I
				\end{bmatrix}
\end{align}

\end{document}

