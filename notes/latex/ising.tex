\documentclass{amsart}
\usepackage[tmargin=1in, bmargin=1in, lmargin=0.8in, rmargin=1in]{geometry}
%%% Sets numbering depth to section level (e.g, no numbered subsections)
\setcounter{secnumdepth}{1}

% include notes style file from Abhishek Shivkumar
\usepackage{macrosabound, theorem-env}

% make font smaller
\usepackage[fontsize=10pt]{fontsize}
% set margins
% only use if strictly necessary, tufte-book calls geometry already
% double geometry call results in "overspecified margins" warning if done as below
% \geometry{left=1in, right=3in, top = 1in, bottom=1in}

%%%%%%%%%%%%%%%%%%
%%%% FOR DEBUGGING
%%%%%%%%%%%%%%%%%%
%\usepackage{layout}
%\usepackage{showframe}

%%% Removes paragraph indentation and changes paragraph line skip

% ensures that the references show up as an unnumbered section
\def\bibsection{\section*{\refname}} 
\begin{document}
%%% The Title and Author only need to be set once at the start of the document. If you take notes for multiple courses in the same document (for example, in a multi-semester sequence for the same course), you can separate the courses with a new Part, and the semester, lecturer, and course only need to be set once at the start of the new course.
\newpage
\title{The Reverse Ising Problem}
\author{Isaac Martin}
\date{Summer 2023}
\maketitle
\section{Introduction and Terminology}
\begin{defn}
  Let $\Sigma = \{-1, +1\}$. An \textbf{Ising circuit} is a function $f:\Sigma^N \to \Sigma^M$ where $N$ and $M$ are finite subsets of $\bN$. For convenience we always assume $N = \{1,...,n\}$ and $M = \{n+1, ..., n+m\}$.
\end{defn}

\section{Pseudo-boolean optimization and polynomial fitting}
A pseudo boolean function (PBF) is any function $f:\{0,1\}\to \bR$. It is a well known fact that any such PBF can be uniquely represented by a multilinear polynomial in $n$ variables [pseudo-boolean optimization Boros, Hammer]; that is, a polynomial
\begin{align*}
  g(x_1,...,x_n) = \sum_{S \subset [n]} a_S \prod_{j \in S}x_j
\end{align*}
with $a_S \in \bR$ which equals $f$ pointwise on $\{0,1\}^n$. To be clear, here $S$ iterates over all subsets of $[n] = \{1,...,n\}$.

It is another well-known fact that the optimization of any pseudo-boolean function can be reduced in polynomial time to an optimization problem on a quadratic polynomial. The original method for accomplishing this was first written by Rosenberg, and since then a reputable zoo of alternative algorithms have been introduced. Most methods share the same basic idea: reduce degree $\geq 3$ monomial terms appearing in the polynomial $g$ by introducing auxiliary variables subject to constraints.

<copy Rosenberg algorithm from Boros, Hammer pg 168>

\begin{thm}\label{thm:rosenberg-reduction}
  Let $f$ be a multilinear polynomial in $n$ variables. There exists an algorithm $\textsc{Reduce}$ which produces a multilinear polynomial $g$ in $n + a$ variables such that
  \begin{align*}
    \min_{(\bfx, \bfa) \in \bB^n \times \bB^a} g(\bfx, \bfa) = \min_{\bfx \in \bB^n} f(\bfx)
  \end{align*}
  and if $(\bfx, \bfa) = \arg\min_{(\bfx, \bfa) \in \bB^n \times \bB^a} g(\bfx, \bfa)$ then $\bfx = \arg\min_{\bfx \in \bB^n} f(\bfx).$
\end{thm}
\begin{proof}
  [Boros Hammer Pseudo Boolean Optimization 2002]
\end{proof}
We need a slightly stronger statement however.
\begin{thm}
  Let $f:\Sigma^N \to \Sigma^M$ be a circuit. Then there exists an Ising circuit with auxiliary spins given by Hamiltonian $H$ which solves $f$.
\end{thm}
\begin{proof}
  Fix $G = N \cup M$ and consider the hamming objective function $\operatorname{ham}:\Sigma^N\times \Sigma^M\to \bR$ defined
  \begin{align*}
    \operatorname{ham}(s,t) = d(t, f(s))
  \end{align*}
  where $d(t, f(s))$ is the Hamming distance between $t$ and the correct output $f(s)$. Then there exists some multilinear polynomial $g$ in $|G|$ variables which recovers $\operatorname{ham}$ pointwise. We now apply Rosenberg reduction to $g$ and set $H$ equal to the terminal quadratic polynomial we obtain. All that remains to show is that on any input level $s$ the output which minimizes $H$ is $f(s)$.

  Fix an input $s$ and suppose that the minimizer of $g^k(s, \cdot)$ has output coordinates $f(s)$. To obtain $g^{k+1}$ we replace some pair $x_ix_j$ by $x_{k+1}$ and add the expression $M(x_ix_j - 2x_ix_{k+1} - 2x_jx_{k+1} + 3x_{k+1})$. Observe that this expression is zero if $x_ix_j = x_{k+1}$ and is strictly positive otherwise. It follows that $g^k(\bfx) = g^{k+1}(\bfx, x_{k+1})$ if $x_{k+1} = x_ix_j$ and $g^k(\bfx) < g^{k+1}(\bfx, x_{k+1})$ if $x_{k+1} \neq x_ix_j$. Hence the minimizer of $g^{k+1}$ on input level $s$ also has the correct output coordinates, and inductively, we conclude that $H$ is an Ising Hamiltonian reproducing the circuit $f$.
\end{proof}

\newpage
\bibliographystyle{abbrvnat}
\bibliography{ising}
\end{document}
